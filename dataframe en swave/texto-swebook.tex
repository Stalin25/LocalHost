\documentclass{report}

\usepackage[spanish]{babel}
\usepackage[utf8]{inputenc}

\usepackage{Sweave}
\begin{document}
\Sconcordance{concordance:texto-swebook.tex:texto-swebook.Rnw:%
1 5 1 1 0 25 1}


\begin{center}\huge{}\textbf{SWEBOK}\end{center}
SWEBOK busca aglutinar en un solo contexto las competencias que debiese tener todo ingeniero de software para desempeñarse competentemente en el mercado. Esta guía organiza el cuerpo de conocimientos en varias áreas específicas de conocimiento tales como:
\begin{enumerate}
  \item {Requerimientos de Software.}
  \item {Diseño de Software.}
  \item {Construcción de software.}
  \item {Pruebas de Software.}
  \item {Mantenimiento de Software.}
  \item {Gestión de la Configuración de Software.}
  \item {Gestión de la Ingeniería de Software.}
  \item {Procesos de la ingeniería de Software.}
  \item {Instrumentos y Métodos de la Ingeniería de Software.}
  \item {Calidad de Software.\\}
\end{enumerate}

Trata los procesos de análisis de requerimientos para detectar y resolver los conflictos emergentes, escribir falencias del sistema y cómo venían interactuar con su medio ambiente.\\ \ \\
El proceso de construcción de software es uno de los pasos más difíciles de la ingeniería de software. Se verificar  dinámica y constantemente la conducta del programa bajo un conjunto finito de casos de prueba y comparar los resultados con lo que se esperaba. Luego se realiza un proceso de mantenimiento de software en el cual sufre una modificación  después de corregir fallas, modificar aspectos visuales u otros aspectos que estén dando problemas en su ejecución. \\ \ \\
La configuración del software permite identificar puntos discretos de tiempo para controlar sus cambios sistemáticamente y a mantener su integridad y trazabilidad a lo largo del ciclo de vida del sistema. Considerando la gestión de ingeniería de software bajo la coordinación de tópicos, estándares de desarrollo e implementación, delimitación del proyecto y equipo de desarrollo.\\ \ \\
El proceso de implementación se centra en los cambios organizacionales, describe la infraestructura, actividades, modelos y consideraciones prácticas de un proceso de implementación y cambios.\\ \ \\
Los métodos de desarrollo determinan una estructura en el desarrollo de software y actividades de mantenimiento con el objetivo de hacer las tareas sistemáticas. Mientras que las herramientas de software están basadas en herramientas de computadora para asistir los procesos de la Ingeniería de Software, a menudo son diseñadas para soportar métodos particulares, su intención es hacer el desarrollo más sistemáticamente.\\ \ \\
Producir un software de calidad es primordial en la satisfacción del cliente. La meta de la ingeniería de software es un producto de calidad. Los procesos de verificación y validación permiten ver la calidad del producto, por lo cual cada producto es individualmente planeado, construido, analizado, medido y mejorado.\\

\end{document}
